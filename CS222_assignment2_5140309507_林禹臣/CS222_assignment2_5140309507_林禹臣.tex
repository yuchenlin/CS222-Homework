\documentclass[12pt,a4paper]{article}
\usepackage{ctex}
\usepackage{amsmath,amscd,amsbsy,amssymb,latexsym,url,bm,amsthm}
\usepackage{epsfig,graphicx,subfigure}
\usepackage{enumitem,balance,mathtools}
\usepackage{wrapfig}
\usepackage{mathrsfs, euscript}
\usepackage[usenames]{xcolor}
\usepackage{hyperref}
%\usepackage{algorithm}
%\usepackage{algorithmi
%\usepackage[vlined,ruled,commentsnumbered,linesnumbered]{algorithm2e}
\usepackage[ruled,vlined,boxed,linesnumbered]{algorithm2e}

\newtheorem{theorem}{Theorem}[section]
\newtheorem{lemma}[theorem]{Lemma}
\newtheorem{proposition}[theorem]{Proposition}
\newtheorem{corollary}[theorem]{Corollary}
\newtheorem{exercise}{Exercise}[section]
\newtheorem*{solution}{Solution}

\renewcommand{\thefootnote}{\fnsymbol{footnote}}

\newcommand{\postscript}[2]
 {\setlength{\epsfxsize}{#2\hsize}
  \centerline{\epsfbox{#1}}}

\renewcommand{\baselinestretch}{1.0}

\setlength{\oddsidemargin}{-0.365in}
\setlength{\evensidemargin}{-0.365in}
\setlength{\topmargin}{-0.3in}
\setlength{\headheight}{0in}
\setlength{\headsep}{0in}
\setlength{\textheight}{10.1in}
\setlength{\textwidth}{7in}
\makeatletter \renewenvironment{proof}[1][Proof] {\par\pushQED{\qed}\normalfont\topsep6\p@\@plus6\p@\relax\trivlist\item[\hskip\labelsep\bfseries#1\@addpunct{.}]\ignorespaces}{\popQED\endtrivlist\@endpefalse} \makeatother
\makeatletter
\renewenvironment{solution}[1][Solution] {\par\pushQED{\qed}\normalfont\topsep6\p@\@plus6\p@\relax\trivlist\item[\hskip\labelsep\bfseries#1\@addpunct{.}]\ignorespaces}{\popQED\endtrivlist\@endpefalse} \makeatother
\begin{document}
\noindent

%========================================================================
\noindent\framebox[\linewidth]{\shortstack[c]{
\Large{\textbf{CS222 Homework 2}}\vspace{1mm}\\
Exercises for Algorithm Design and Analysis by Li Jiang, 2016 Autumn Semester\\
5140309507 林禹臣 yuchenlin@sjtu.edu.cn}}

~\\
\begin{enumerate}

\item There are two sorted arrays nums1 and nums2 of size m and n respectively.

Find the median of the two sorted arrays. The overall run time complexity should be O(log (m+n)).

Example 1:

nums1 = [1, 3]

nums2 = [2]

The median is 2.0

Example 2:

nums1 = [1, 2]

nums2 = [3, 4]

The median is (2 + 3)/2 = 2.5

Input:

int nums1[]; int m;

int nums2[]; int n;

Output:

double median.

%Your answer should be written here.


\begin{solution}  
My solution here is based on a divide-and-conquer idea so that it can be a $O(log(m+n))$ algorithm. 


The most naive intuition for solving this problem is first to merge sort all m+n elements and then calculate the median. But this will cause an $O(m+n)$ time complexity. The reason why this intuition is a bad idea is that it does not consider very much about how to avoid many unnecessary comparison.


We now consider a more general problem: \emph{find the k-th smallest element of the union of this two sorted arrays.}





\emph{\bf{Proposition: If $a+b=k (a,b>0)$ and $nums1[a-1] < nums2[b-1]$, we can say that the k-th smallest element ($T$) must be not in $nums1[0:a]$.}}

By contradiction, if $T$ is in nums1[0:a], then $nums1[a-1]=T$, since $a < k$. Also, because $a+b=k$, $T$ is the maximum value of $\{nums1[0:a-1],~nums2[0:b-1]\}$, and thus \emph{\underline{$T>=nums2[b-1]$}}. However, it causes a conflict that \emph{\underline{$T = nums1[a-1] < nums2[b-1]$}}.


Algorithm~\ref{alg:alg1}  and Algorithm~\ref{alg:alg2} show the process. Evidently, the time complexity is $O(log(\frac{m+n}{2}))$.



\begin{algorithm} 
  \label{alg:alg1}
  \caption{$findK$: Find the k-th smallest element of the union of the two sorted arrays.} 
  \KwIn{int $nums1[]$, int $m$, int $nums2[]$, int $n$, int $k$ } 
  \KwOut{double $k$-th smallest element}


  \underline{\emph{// ensure that $m \le n$ }}

  \If{$m>n$}{
  	\Return{findK(~$nums2[:]$,~$n$,~$nums1[:]$,~$m$,~$k$~);}
  }


  \underline{\emph{// two base cases }}


  \If{$m==0$}{
  	\Return{$nums2[k-1]$;}
  }

  \If{$k==1$}{
  	\Return{$min\{nums1[0],nums2[0]\}$;}
  }

  \underline{\emph{// ensure that $index\_1 + index\_2 == k$ and $index\_2 >= index\_1$}}


  $index\_1 = min\{k/2 , m\} ;$


  $index\_2 = k - index\_1 ;$  

  \underline{\emph{// reduce the problem to a half-smaller one }}


  \If{$nums1[index\_1-1] < nums2[index\_2-1]$}{
  	\Return{findK(~$nums1[index\_1:]$,~$m-index\_1$,~$nums2[:]$,~$n$,~$index\_2$~);}
  }\ElseIf{$nums1[index\_1-1] > nums2[index\_2-1]$}{
  	\Return{findK(~$nums1[:]$,~$m$,~$nums2[index\_2:]$,~$n-index\_2$,~$index\_1$~);}
  }\Else{
  	\Return{$nums1[index\_1]$}
  }

\end{algorithm}
\begin{algorithm} 
  \label{alg:alg2}
  \caption{$findMedian$: Find the median of the union of the two sorted arrays.} 
  \KwIn{int $nums1[]$,int $m$, int $nums2[]$,int $n$} 
  \KwOut{double the median}

  $ all = m + n ;$
  \If{all is even}{
  	\Return $findK(~nums1,~m,~nums2,~n~,(all/2)+1);$
  }\Else{
  	\Return $(~findK(~nums1,~m,~nums2,~n~,all/2) + findK(~nums1,~m,~nums2,~n~,(all/2)+1)~)~/~2$
  }
\end{algorithm}


\end{solution}

~\\
~\\


\pagebreak
\item Find the contiguous subarray within an array (containing at least one number) which has the largest sum.

For example, given the array [-2,1,-3,4,-1,2,1,-5,4], the contiguous subarray [4,-1,2,1] has the largest sum = 6.

Input:

int A[]: the input array.

int N: length of A.

Output:

return the largest sum.

%Your answer should be written here.

\begin{solution}
I have two solutions here. The first one is using Divide and Conquer strategy, which gives us an $O(nlgn)$ time complexity. While the other is an online solution, which means its time complexity is just $O(n)$.

\emph{\bf{Solution 1: Divide and Conquer }}

As usual, we have 3 steps when we are using this strategy. Say we are computing the largest sum from $x$ to $y$ of array $A[]$.

First, we compute the large sum of its left half part by calling this function with $A,x,\frac{(x+y)}{2}-1$. Suppose the result is $L$. Second, we do the same thing to the right half part by calling the function with $A,\frac{(x+y)}{2},y$ and say the result is $R$. Third, we find the largest sum of the sub-array including the middle node. Say the result is $M$.

Then we can return the maximum of $L$,$R$ and $M$. See Algorithm~\ref{alg:alg3}. We can simply call $getLargestSum(A,~0,~N)$ and get the result. The time complexity of this solution is evidently $O(nlgn)$ since $T(n)=2T(\frac{n}{2})+O(n)$


\begin{algorithm} 
  \label{alg:alg3}
  \caption{$getLargestSum$: Find the largest sum of a contiguous sub-array of $A[x:y)$.} 
  \KwIn{int $A[]$,int $x$, int $y$} 
  \KwOut{int the largest sum}

  \If{$y-x==1$}{
  	\Return A[x];
  }
  $mid = [(x+y)/2];$\\
  $L = getLargestSum(A,~x,~mid);$\\
  $R = getLargestSum(A,~mid,~y);$\\

  $tmp = 0;$\\
  $Ml = 0;$\\
  \For{$i$ from $mid-1$ to $x$}{
  	$tmp+=A[i];$\\
  	$Ml = max\{Ml,~tmp\}$
  }

  $tmp = 0;$\\
  $Mr = 0;$\\
  \For{$i$ from $mid$ to $y-1$}{
  	$tmp+=A[i];$\\
  	$Mr = max\{Mr,~tmp\}$;\\
  }
  $M = Ml + Mr;$\\
  
  \Return $max\{L,~R~,M\};$
\end{algorithm}
~\\
~\\


\emph{\bf{Solution 2: Online Processing Algorithm }}\\
Actually, we have an intuition that if we calculate the sum from the first node to the last one, we can reset the sum to 0 every time when our current sum becomes negative. It is because that if we go on, then we can say why not we just keep the next one( say $P$)? No matter what number $P$ is, $P$ must be greater than the $currentSum+P$ when $currentSum$ is negative. Thus, we can have a most simple and most efficient($O(n)$) algorithm. See Algorithm~\ref{alg:alg4}.

\begin{algorithm} 
  \label{alg:alg4}
  \caption{$getLargestSum$: Find the largest sum of a contiguous sub-array of $A[0:N)$.} 
  \KwIn{int $A[]$,int $N$} 
  \KwOut{int the largest sum}

  $currentSum = 0;$\\
  $maxSum= 0;$\\
  \For{$i \in [~0,N~) $ }{
  	$currentSum += A[i];$\\
  	$maxSum = max\{maxSum,~currentSum\}$;\\
  	\If{$currentSum<0$}{
  		$currentSum = 0;$\\
  	}
  }
  \Return $maxSum;$
\end{algorithm}
% int MaxSubseqSum4( int A[], int N )  
% {   int ThisSum, MaxSum;
%     int i;
%     ThisSum = MaxSum = 0;
%     for( i = 0; i < N; i++ ) {
%           ThisSum += A[i]; /* 向右累加 */
%           if( ThisSum > MaxSum )
%                   MaxSum = ThisSum; /* 发现更大和则更新当前结果 */
%           else if( ThisSum < 0 ) /* 如果当前子列和为负 */
%                   ThisSum = 0; /* 则不可能使后面的部分和增大,抛弃之 */  
%     }
%     return MaxSum;  
% }

\end{solution}

\item Given a non-empty array containing only positive integers, find if the array can be partitioned into two subsets such that the sum of elements in both subsets is equal.

Note:

Each of the array element will not exceed 100.

The array size will not exceed 200.

Example 1:

Input: [1, 5, 11, 5]

Output: true

Explanation: The array can be partitioned as [1, 5, 5] and [11].

Example 2:

Input: [1, 2, 3, 5]

Output: false

Explanation: The array cannot be partitioned into equal sum subsets.

Input:

int A[]: the input array.

int N: length of A.

Output:

return true or false.

%Your answer should be written here.
\begin{solution}
I cannot figure out a good Divide-and-Conquer Strategy for this problem. However, from my historical programming experience, this problem is a typical Dynamic Programming problem. Thus, I tried to implement a DP solution.\\
First, we must know that the sum of the array which is very useful. If the sum $S$ is odd then we can simply return False. If S is even, then the problem would be transformed to a typical problem that whether we can select some certain elements from the array so that the sum of them is exactly $S/2$.\\
Using dynamic programming strategy means we need to define a $state$ first.
Here we say a state $f(i,j)$ means if our bag's capacity is $j$ kg and we can select from $i$ items then we can at most fill the bag with $f(i,j)$ kg. ( In this sense, $A[i]$ means the weight of item $i$.)\\
Thus, we can just return whether $f(N,S/2) == S/2$; If it can be, then that is to say we can select some certain items from the all N items and make their sum is exactly $S/2$.

Here comes the state transform function: \\
$f(0,0) = 0;$\\
$f(i,j) = max\{~ f(i-1,j),~ f( i-1, j-A[i])+A[i] \};$\\

See Algorithm~\ref{alg:alg5}. Evidently, the time complexity is $O(S*N)$ where $S$ is the sum of these numbers. 

\begin{algorithm} 
  \label{alg:alg5}
  \caption{$judgeEqual$} 
  \KwIn{int $A[]$,int $N$} 
  \KwOut{boolean whether we can separate the set accordingly}
  $S =$ the sum of all the elements;\\
  \If{S is odd}{
  	\Return $False$;
  }
  $f = new~int[N][S/2+1];$\\

  \For{$i \in [1,~n)$ }{
  	\For{$j \in [A[i],~S/2]$ }{
  		$f(i,j) = max\{~ f(i-1,j),~ f( i-1, j-A[i])+A[i] \};$\\
  	}
  }
  \Return $f(N,S/2) == S/2;$
\end{algorithm}

Another more simple and efficient algorithm is to consider this problem as a typical 0-1 bagging problem and use bit-set to solve the problem. We define the state $f(i)=1$ means that i can be a sum of a certain subset of original set. According to the description, we can know that the largest sum will be $200*100 = 40000$, and thus the target will be up to $20000+1$. Then our goal is to find whether f(target)==1 or not.\\
We can use left-shift as a plus on all states and use bit-or to save all the changes.\\
For example, ${1,2,5,2}$, firstly the bitset will be 000001,and then when we consider 1, we plus 1 on all the positive numbers and save the changes, so the bitset becomes 000011, and then 001111,101111(which means 5 can be generated and we can stop here.). See Algorithm~\ref{alg:alg6}.
If we consider left-shift can be a basic operation, then the time complexity will be $O(n)$;\\

\begin{algorithm} 
  \label{alg:alg6}
  \caption{$judgeEqualWithBitset$} 
  \KwIn{int $A[]$,int $N$} 
  \KwOut{boolean whether we can separate the set accordingly}
  $S =$ the sum of all the elements;\\
  \If{S is odd}{
  	\Return $False$;
  }
  $f = bitset(200001);$\\
  $f(0)=1;$\\
  \For{$i \in [0,~n)$ }{
  	$f = f~|~(f<<A[i]);$\\
  }
  \Return $f(S/2) == 1;$
\end{algorithm}
\end{solution}

~\\
~\\

\end{enumerate}
%========================================================================
\end{document}
