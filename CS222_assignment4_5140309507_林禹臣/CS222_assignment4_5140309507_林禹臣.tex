\documentclass[12pt,a4paper]{article}
\usepackage{ctex}
\usepackage{amsmath,amscd,amsbsy,amssymb,latexsym,url,bm,amsthm}
\usepackage{epsfig,graphicx,subfigure}
\usepackage{enumitem,balance,mathtools}
\usepackage{wrapfig}
\usepackage{mathrsfs, euscript}
\usepackage[usenames]{xcolor}
\usepackage{hyperref}
%\usepackage{algorithm}
%\usepackage{algorithmic}
%\usepackage[vlined,ruled,commentsnumbered,linesnumbered]{algorithm2e}
\usepackage[ruled,lined,boxed,linesnumbered]{algorithm2e}

\newtheorem{theorem}{Theorem}[section]
\newtheorem{lemma}[theorem]{Lemma}
\newtheorem{proposition}[theorem]{Proposition}
\newtheorem{corollary}[theorem]{Corollary}
\newtheorem{exercise}{Exercise}[section]
\newtheorem*{solution}{Solution}

\renewcommand{\thefootnote}{\fnsymbol{footnote}}

\newcommand{\postscript}[2]
 {\setlength{\epsfxsize}{#2\hsize}
  \centerline{\epsfbox{#1}}}

\renewcommand{\baselinestretch}{1.0}

\setlength{\oddsidemargin}{-0.365in}
\setlength{\evensidemargin}{-0.365in}
\setlength{\topmargin}{-0.3in}
\setlength{\headheight}{0in}
\setlength{\headsep}{0in}
\setlength{\textheight}{10.1in}
\setlength{\textwidth}{7in}
\makeatletter \renewenvironment{proof}[1][Proof] {\par\pushQED{\qed}\normalfont\topsep6\p@\@plus6\p@\relax\trivlist\item[\hskip\labelsep\bfseries#1\@addpunct{.}]\ignorespaces}{\popQED\endtrivlist\@endpefalse} \makeatother
\makeatletter
\renewenvironment{solution}[1][Solution] {\par\pushQED{\qed}\normalfont\topsep6\p@\@plus6\p@\relax\trivlist\item[\hskip\labelsep\bfseries#1\@addpunct{.}]\ignorespaces}{\popQED\endtrivlist\@endpefalse} \makeatother
\begin{document}
\noindent

%========================================================================
\noindent\framebox[\linewidth]{\shortstack[c]{
\Large{\textbf{CS222 Homework 4}}\vspace{1mm}\\
Exercises for Algorithm Design and Analysis by Li Jiang, 2016 Autumn Semester}}
~\\
\begin{enumerate}

\item Given a non-empty integer array, find the minimum number of moves required to make all array elements equal, where a move is incrementing a selected element by 1 or decrementing a selected element by 1.

You may assume the array's length is at most 10,000.

Example:

Input:
[1,2,3]

Output:
2

Explanation:

Only two moves are needed (remember each move increments or decrements one element):

$[1,2,3] => [2,2,3] => [2,2,2]$

Input:

int A[]: the input array.

int N: length of A.

Output:

int minMoves.

%Your answer should be written here.

~\\
~\\


\item Given a string that consists of only uppercase English letters, you can replace any letter in the string with another letter at most k times. Find the length of a longest substring containing all repeating letters you can get after performing the above operations.
    
Note:Both the string's length and k will not exceed 104.

Example:

Input:

s = "AABABBA", k = 1

Output:

4

Explanation:

Replace the one 'A' in the middle with 'B' and form "AABBBBA".
The substring "BBBB" has the longest repeating letters, which is 4.

Input:

string s;

int k;

Output:

return the length of the longest substring.

%Your answer should be written here.

~\\
~\\


\item You are given an array x of n positive numbers. You start at point (0,0) and moves x[0] metres to the north, then x[1] metres to the west, x[2] metres to the south, x[3] metres to the east and so on. In other words, after each move your direction changes counter-clockwise.

Write a one-pass algorithm with O(1) extra space to determine, if your path crosses itself, or not.

Example 1:

Given x = [2, 1, 1, 2]

Return true (self crossing)

Example 2:

Given x = [1, 2, 3, 4]

Return false (not self crossing)

Example 3:

Given x = [1, 1, 1, 1]

Return true (self crossing)

Input:

int x[]: the input array.

int N: length of x.

Output:

return true or false.

%Your answer should be written here.

~\\
~\\

\end{enumerate}
%========================================================================
\end{document}
