\documentclass[12pt,a4paper]{article}
\usepackage{ctex}
\usepackage{amsmath,amscd,amsbsy,amssymb,latexsym,url,bm,amsthm}
\usepackage{epsfig,graphicx,subfigure}
\usepackage{enumitem,balance,mathtools}
\usepackage{wrapfig}
\usepackage{mathrsfs, euscript}
\usepackage[usenames]{xcolor}
\usepackage{hyperref}
%\usepackage{algorithm}
%\usepackage{algorithmi
%\usepackage[vlined,ruled,commentsnumbered,linesnumbered]{algorithm2e}
\usepackage[ruled,vlined,boxed,linesnumbered]{algorithm2e}

\newtheorem{theorem}{Theorem}[section]
\newtheorem{lemma}[theorem]{Lemma}
\newtheorem{proposition}[theorem]{Proposition}
\newtheorem{corollary}[theorem]{Corollary}
\newtheorem{exercise}{Exercise}[section]
\newtheorem*{solution}{Solution}

\renewcommand{\thefootnote}{\fnsymbol{footnote}}

\newcommand{\postscript}[2]
 {\setlength{\epsfxsize}{#2\hsize}
  \centerline{\epsfbox{#1}}}

\renewcommand{\baselinestretch}{1.0}

\setlength{\oddsidemargin}{-0.365in}
\setlength{\evensidemargin}{-0.365in}
\setlength{\topmargin}{-0.3in}
\setlength{\headheight}{0in}
\setlength{\headsep}{0in}
\setlength{\textheight}{10.1in}
\setlength{\textwidth}{7in}
\makeatletter \renewenvironment{proof}[1][Proof] {\par\pushQED{\qed}\normalfont\topsep6\p@\@plus6\p@\relax\trivlist\item[\hskip\labelsep\bfseries#1\@addpunct{.}]\ignorespaces}{\popQED\endtrivlist\@endpefalse} \makeatother
\makeatletter
\renewenvironment{solution}[1][Solution] {\par\pushQED{\qed}\normalfont\topsep6\p@\@plus6\p@\relax\trivlist\item[\hskip\labelsep\bfseries#1\@addpunct{.}]\ignorespaces}{\popQED\endtrivlist\@endpefalse} \makeatother
\begin{document}
\noindent

%========================================================================
\noindent\framebox[\linewidth]{\shortstack[c]{
\Large{\textbf{CS222 Homework 4}}\vspace{1mm}\\
Exercises for Algorithm Design and Analysis by Li Jiang, 2016 Autumn Semester\\
5140309507 林禹臣 yuchenlin@sjtu.edu.cn}}


~\\
\begin{enumerate}

\item Given a non-empty integer array, find the minimum number of moves required to make all array elements equal, where a move is incrementing a selected element by 1 or decrementing a selected element by 1.

You may assume the array's length is at most 10,000.

Input:

int A[]: the input array.

int N: length of A.

Output:

int minMoves.

%Your answer should be written here.

\begin{solution}
Let us formulate this problem first. Function $F(T)$ to indicate the total movements when the final equal number is $T$. So we have:\\
\begin{equation*}
F(T)= \sum_{i=1}^{N}|x_i-T|
\end{equation*}

The problem is to find the minimum of the $F(T)$. This reminds me of one of the most important properties of median: \begin{bf}{Median minimizes the sum of Absolute Deviations}. \end{bf} I would like to prove it later. Now if we all agree with the statement, what we should do next is very simple:\\\\
\emph{Step 1.} Find the median $M$ of the array.\\
\emph{Step 2.} Calculate $F(M)$.\\
\emph{Step 3.} Return $F(M)$ as the final result.\\

The time complexity of \emph{Step 1} can be $O(n)$ with a divide-and-conquer strategy, which I have mentioned in the \emph{Assignment 2}. The time complexity of \emph{Step 2} is obviously $O(n)$ and \emph{Step 3} is $O(1)$. Therefore, the total time complexity is $O(n)$.

Now, let us prove the above-mentioned statement that Median minimizes the sum of Absolute Deviations.\\
First of all, I would like to point out some wrong or weak statements. Some say that we must choose the final $T$ from the elements in the array to minimize the $F(T)$, which is definitely wrong for that if the length is even and then we can choose any number between the middle two elements as the final T. Also, some say that choosing average can cause wrong output, then median is the optimal choice. It is just a guess, not a mathematical proof. \\
\pagebreak
\\\\

\begin{bf}Proof 1.\end{bf} Here, I would firstly use derivatives to illustrate why median is optimal choice and provide another proof with basic maths.
We all know:
\begin{equation*}
	\frac{d|x|}{dx}=sgn(x)
\end{equation*}
Thus,
\begin{equation*}
	\frac{dF}{dx}=\sum_{i=1}^{N} sgn(x_i)
\end{equation*}
We should notice that in derivative is $0$ only if the number of positive elements is equal to the number of negative elements. Meanwhile, $x=median$ can make sure that the number of elements which are less than median and the number of elements which are greater than median is equal. Plus, if $x<median$ and then the derivative is negative; if $x>median$ and then the derivative is positive.\\Thus, median is an optimal solution.\\

\begin{bf}Proof 2.\end{bf}  is as follows: 
Suppose the length is odd and $T \le M$.
We can conclude that:\\
\begin{equation*}
x_1 \le x_2 \le ... \le x_s\le... \le T \le...\le x_{t-1} \le x_t = M \le x_{t+1} \le ...\le x_N ~~~(t=\frac{n-1}{2})
\end{equation*}
\begin{flalign}
\begin{split}
&F(T)= \sum_{i=1}^{N}(x_i-T)\\ 
&= \sum_{i=1}^{s}(T-x_i) + \sum_{i=s+1}^{N}(x_i-T) \\
&= [\sum_{i=1}^{t-1}(T-x_i) - \sum_{i=s+1}^{t-1}(T-x_i)] + [\sum_{i=s+1}^{t-1}(x_i-T) + 0 + \sum_{i=t+1}^{N}(x_i-T) ]\\
&= [\sum_{i=1}^{t-1}(T-M+M-x_i) - \sum_{i=s+1}^{t-1}(T-x_i)] + [\sum_{i=s+1}^{t-1}(x_i-T) + \sum_{i=t+1}^{N}(x_i-M+M-T)]\\
&= \sum_{i=1}^{t-1}(M-x_i) + (t-1)(T-M) + 2\sum_{i=s+1}^{t-1}(x_i-T) + \sum_{i=t+1}^{N}(x_i-M)+(n-t)(M-T)\\
&= \sum_{i=1}^{t-1}(M-x_i) +  \sum_{i=t+1}^{N}(x_i-M) + 2\sum_{i=s+1}^{t}(x_i-T) + (n-2t+1)(M-T)\\
&= \sum_{i=1}^{t-1}(M-x_i) +  \sum_{i=t+1}^{N}(x_i-M) + 2\sum_{i=s+1}^{t}(x_i) - 2(t-s)*T\\
\end{split}&
\end{flalign}
Thus, if we want to minimize the $F(T)$, we just need to maximize $T$, which means when $T=M$ we can get the minimum of $F(T)$.It is similar when $T \ge M$ and we can insert a certain number between the middle two numbers to make the array odd. \\
\end{solution}
\pagebreak

~\\
~\\


\item Given a string that consists of only uppercase English letters, you can replace any letter in the string with another letter at most k times. Find the length of a longest substring containing all repeating letters you can get after performing the above operations.
    
Note:Both the string's length and k will not exceed 104.

Example:

Input:

s = "AABABBA", k = 1

Output:

4

Explanation:

Replace the one 'A' in the middle with 'B' and form "AABBBBA".
The substring "BBBB" has the longest repeating letters, which is 4.

Input:

string s;

int k;

Output:

return the length of the longest substring.

%Your answer should be written here.

~\\
~\\


\item You are given an array x of n positive numbers. You start at point (0,0) and moves x[0] metres to the north, then x[1] metres to the west, x[2] metres to the south, x[3] metres to the east and so on. In other words, after each move your direction changes counter-clockwise.

Write a one-pass algorithm with O(1) extra space to determine, if your path crosses itself, or not.

Example 1:

Given x = [2, 1, 1, 2]

Return true (self crossing)

Example 2:

Given x = [1, 2, 3, 4]

Return false (not self crossing)

Example 3:

Given x = [1, 1, 1, 1]

Return true (self crossing)

Input:

int x[]: the input array.

int N: length of x.

Output:

return true or false.

%Your answer should be written here.

~\\
~\\

\end{enumerate}
%========================================================================
\end{document}
